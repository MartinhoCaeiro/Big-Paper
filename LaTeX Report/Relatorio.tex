\documentclass[a4paper]{article}
% Pacotes necessários
\usepackage[portuguese]{babel}
\usepackage[backend=biber, style=apa, citestyle=apa, language=portuguese]{biblatex}
\usepackage{csquotes}
\addbibresource{Recursos/referencias.bib}

\usepackage{amsmath}
\usepackage{graphicx}
\usepackage{subcaption}
\usepackage{setspace}
\usepackage{siunitx} % Required for alignment
\sisetup{
  round-mode          = places, % Rounds numbers
  round-precision     = 2, % to 2 places
}
\usepackage{enumerate}
\usepackage{enumitem}
\usepackage{amsmath}
\usepackage{karnaugh-map}
\usepackage[section]{placeins}
\usepackage{geometry}
\usepackage{amssymb}
\usepackage{titling}
\usepackage[T1]{fontenc}
\usepackage{float}
\usepackage[hidelinks]{hyperref}
\usepackage{xcolor}
\usepackage{indentfirst}
\usepackage{array}
\usepackage{soul}
\usepackage{afterpage}
\newcolumntype{P}[1]{>{\centering\arraybackslash}p{#1}}
\onehalfspacing

% Comando para criar uma página vazia
\newcommand\myemptypage{
    \null
    \thispagestyle{empty}
    \addtocounter{page}{-1}
    \newpage
}

% Página de título principal
\newcommand{\firsttitlepage}{
    \begin{titlepage}
        \centering
        \vspace*{1cm}
        
        % Logos superior
        \begin{figure}[h!]
            \centering
            \includegraphics[width=6cm]{Recursos/LOGO_IPB} % Substitua pelo caminho da imagem
            \vspace{0.5cm}
        \end{figure}

        % Informações da instituição
        \large\textbf{INSTITUTO POLITÉCNICO DE BEJA} \\
        \large\textbf{Escola Superior de Tecnologia e Gestão} \\
        \large\textbf{Mestrado em Engenharia de Segurança Informática} \\
        \large\textbf{Direito na Segurança Informática e no Cibercrime} \\
        
        \vspace{2cm}
        
        % Título do projeto
        {\Huge \textbf{Acórdão de 22 de junho de 2021}} \\
        
        \vspace{1.5cm}
        
        % Autores
        \large Martinho José Novo Caeiro - 23917 \\
        
        \vfill
        
        % Logo inferior
        \begin{figure}[h!]
            \centering
            \includegraphics[width=6cm]{Recursos/IPBejaESTIG.jpg} % Substitua pelo caminho da imagem
        \end{figure}
        
        \vspace{1cm}
        
        % Local e data
        {\large Beja, janeiro de 2026}
    \end{titlepage}
}

\newcommand{\secondtitlepage}{
    \begin{titlepage}
        \centering
        \vspace*{1cm}
        
        % Informações da instituição
        \large\textbf{INSTITUTO POLITÉCNICO DE BEJA} \\
        \large\textbf{Escola Superior de Tecnologia e Gestão} \\
        \large\textbf{Mestrado em Engenharia de Segurança Informática} \\
        \large\textbf{Direito na Segurança Informática e no Cibercrime} \\
        
        \vspace{2cm}
        
        % Título do projeto
        {\Huge \textbf{Acórdão de 22 de junho de 2021}} \\
        
        \vspace{1.5cm}
        
        % Autores
        \large Martinho José Novo Caeiro - 23917 \\

        \vspace{2cm}

        % Orientador
        \large Orientador: Manuel David Rodrigues Masseno \\
        
        \vfill
        
        % Local e data
        {\large Beja, janeiro de 2026}
    \end{titlepage}
}

\begin{document}


\pagenumbering{gobble} % Oculta numeração da página

% Primeira página de título
\firsttitlepage

\secondtitlepage

\vspace{1cm}
\newpage
%--------------------------------------------------------------------------------------------------------------------------------------

\vspace{1cm}

% Início do conteúdo do relatório
\newpage
\doublespacing
\doublespacing

\newpage
\pagenumbering{arabic}

\section{O Acórdão}\label{intro}
O acordão em questão trata-se de dois processos, C‑682/18, onde Frank Peterson, produtor de música, 
colocou o YouTube e a sua representante legal, a Google, nos tribunais alemães devido à distribuição
no YouTube, em 2008, de vários fonogramas sobre os quais, alegadamente, 
Peterson detém vários direitos (\cite{acordao20210622}).

E C‑683/18, onde Elsevier colocou Cyando nos tribunais alemães devido à distribuição 
na sua plataforma de armazenagem e de partilha de ficheiros «Uploaded», em 2013, 
de várias obras sobre as quais a Elsevier detém direitos exclusivos.

O Bundesgerichtshof (Supremo Tribunal Federal Alemão), tomou conhecimento destes dois processos e, 
submeteu várias questões prejudiciais ao Tribunal de Justiça para que este determine, 
a responsabilidade dos operadores de plataformas em linha quando estejam em causa obras protegidas 
pelos direitos de autor que são colocadas nessas plataformas, de forma ilícita, pelos seus utilizadores.\\

Nos próximos tópicos será feita uma análise do acórdão, de modo a encontrar o porquê das decisões tomadas pelo 
Tribunal de Justiça da União Europeia, baseando-se no teor do acórdão do mesmo Tribunal nos processos C-682/18 e C-683/18, 
nas disposições relevantes das Diretivas 2001/29/CE, 2000/31/CE e 2019/790, bem como em contributos doutrinais selecionados 
sobre responsabilidade das plataformas e direitos de autor, não tendo sido utilizados outros recursos externos além destes.

%---------------------------------------------------------------------------------------------------------------------------
\newpage
\section{Análise Jurídica}\label{con}

%---------------------------------------------------------------------------------------------------------------------------
\subsection{Com atos da época}
A primeira pergunta feita pelo Bundesgerichtshof é se o operador de uma plataforma de partilha de vídeos
ou de uma plataforma de armazenagem e de partilha de ficheiros é responsável quando os utilizadores colocam
ilegalmente à disposição do público conteúdos protegidos.\\

Assim sendo, temos o facto de que os autores têm direito a autorizar ou proibir a utilização das suas obras,
abrangido no Artigo 3.º da Diretiva 2001/29/CE (\cite{diretiva200129-1}), onde se confere aos titulares de direitos
o controlo sobre os atos de comunicação ao público, mas que não impõe, por si só, uma obrigação geral direta
aos operadores de plataformas, sendo necessário apurar se estes próprios praticam um ato de
“comunicação ao público”, na aceção desta disposição. Tal interpretação é reforçada por Hanuz (\cite{hanuz2020_direct_copyright}), 
que analisa a responsabilidade direta das plataformas frente a conteúdos gerados pelos utilizadores.\\

Neste contexto, o Tribunal de Justiça recorda que o conceito de “comunicação ao público” exige uma
intervenção indispensável e deliberada do operador, retomando a sua jurisprudência anterior,
nomeadamente os acórdãos VG Bild-Kunst, GS Media e Stichting Brein (n.os 62-68 do acórdão).
Com base nesses critérios, decidiu que, em regra, os operadores em questão não realizam um
“ato de comunicação ao público” quando os utilizadores carregam conteúdos ilegais,
salvo se tiverem um papel ativo e deliberado na disponibilização desses conteúdos,
que lhes permita ter conhecimento ou controlo dos mesmos (n.os 84-86 do acórdão), 
conforme discutido também por Asadi (\cite{asadi2023_digital_platforms_ip}), que evidencia a diferença entre responsabilidade indireta e direta nas plataformas.

\newpage
Não só os autores têm direitos exclusivos, como também têm o direito de obter injunções contra os
intermediários cujos serviços foram usados para violar os direitos de autor, conforme o
Artigo 8.º, n.º 3, da mesma Diretiva.
O Tribunal de Justiça sublinha que esta disposição permite aos titulares de direitos obter medidas
judiciais contra plataformas que não atuem após terem conhecimento de conteúdos ilícitos
(n.os 103-106 do acórdão), o que significa que, se o autor descobrir que o operador não está a respeitar
o seu pedido, pode recorrer aos tribunais para fazer valer os seus direitos, reforçando a análise de Romero (\cite{romero2024_online_platforms_copyright}) sobre o equilíbrio entre direitos dos autores e atuação das plataformas.\\

Perante estes dois artigos, tal como interpretados pelo Tribunal de Justiça, conclui-se que os autores
têm meios legais para proteger os seus direitos de autor e para poderem agir judicialmente
quando necessário, sem que isso implique automaticamente uma responsabilidade direta
dos operadores por todos os conteúdos colocados pelos utilizadores, uma conclusão também corroborada por Reis et al. (\cite{reis2025_responsabilidade_plataformas}).\\\\

%---------------------------------------------------------------------------------------------------------------------------
A segunda pergunta revolve em torno da possibilidade de os operadores tirarem proveito do facto de que,
por norma, não têm responsabilidade automática pelo conteúdo colocado na plataforma.\\

Responde-se a esta pergunta com o Artigo 14.º da Diretiva 2000/31/CE (\cite{diretiva200031-1}),
tal como interpretado pelo Tribunal de Justiça nos n.os 101-109 do acórdão, onde é referido que o operador
beneficia de uma isenção de responsabilidade pelo conteúdo colocado pelos utilizadores,
desde que não tenha conhecimento efetivo do conteúdo ilegal
ou que atue rapidamente para remover ou bloquear o acesso ao conteúdo ilegal
assim que tenha conhecimento do mesmo. Este entendimento é reforçado por Hanuz (\cite{hanuz2020_direct_copyright}) e Asadi (\cite{asadi2023_digital_platforms_ip}), 
que destacam a distinção entre responsabilidade indireta (quando apenas hospeda conteúdo) e responsabilidade direta (quando há controlo ou conhecimento ativo).\\

\newpage
Este critério de “conhecimento” não está isento de zonas cinzentas,
uma vez que inclui não apenas o conhecimento efetivo, mas também o conhecimento de factos
ou circunstâncias que tornem a ilegalidade manifesta, como resulta do n.º 104 do acórdão. Romero (\cite{romero2024_online_platforms_copyright}) 
reforça esta interpretação, mostrando que a jurisprudência europeia procura equilibrar a proteção dos direitos de autor com a liberdade de expressão e operação das plataformas.\\

Mas os operadores têm ainda meios para se protegerem, como decorre do Artigo 15.º da mesma Diretiva,
onde é referido que não estão obrigados a proceder a uma monitorização geral dos conteúdos
colocados pelos utilizadores, salvo se existir uma ordem judicial nesse sentido,
o que o Tribunal de Justiça reafirma no n.º 112 do acórdão. Reis et al. (\cite{reis2025_responsabilidade_plataformas}) destacam que, mesmo sem monitorização obrigatória, 
as plataformas podem ser responsabilizadas se ignorarem avisos claros de conteúdos ilegais.\\

Perante estes dois artigos, tal como interpretados pelo Tribunal de Justiça, conclui-se que os operadores
não são automaticamente responsáveis pelo conteúdo colocado pelos utilizadores,
nem estão obrigados a ser proativos na sua fiscalização,
mas devem agir rapidamente quando tenham conhecimento da existência de conteúdo ilegal, uma conclusão que encontra respaldo nos estudos de Hanuz, Asadi e Romero citados acima.\\

Dados todos os factos, conclui-se que, mesmo sem a obrigatoriedade de os operadores monitorizarem
os conteúdos, os autores têm todo o direito de pedir a remoção do conteúdo ilegal
e de recorrer a injunções judiciais, não existindo conflito entre as diretivas aplicáveis,
tal como interpretadas pelo Tribunal de Justiça no acórdão em análise.

%---------------------------------------------------------------------------------------------------------------------------
\newpage
\subsection{Com atos correntes}

Ambas as Diretivas tiveram alterações, mais especificamente em 2017 e 2019 para a Diretiva 2001/29/CE
(\cite{diretiva200129-2}) e em 2024 para a Diretiva 2000/31/CE (\cite{diretiva200031-2}).\\

Em especial, temos a Diretiva 2019/790 (\cite{diretiva2019790}), que introduziu no seu Artigo 17.º
um novo regime jurídico, passando a considerar que determinados prestadores de serviços
de partilha de conteúdos em linha realizam, eles próprios, “atos de comunicação ao público”
ou de colocação à disposição do público quando disponibilizam conteúdos carregados
pelos seus utilizadores.

Consequentemente, estas plataformas passam a ser, em princípio, diretamente responsáveis
pelos conteúdos protegidos por direitos de autor,
salvo se demonstrarem ter envidado os melhores esforços para obter autorizações
dos titulares de direitos e para impedir a disponibilização de obras não autorizadas.\\

No entanto, este novo regime não é aplicável retroativamente,
razão pela qual os processos C-682/18 e C-683/18 não tiveram a sua decisão alterada,
tendo o Tribunal de Justiça aplicado o direito da União em vigor à data dos factos.

Assim, apesar de a Diretiva 2019/790 alterar profundamente o regime de responsabilidade
das plataformas para o futuro, tal não teve impacto direto na solução jurídica adotada
pelo Tribunal de Justiça no acórdão em análise, mantendo-se válidas, para estes casos concretos,
as conclusões retiradas anteriormente.\\

Caso estes processos fossem julgados atualmente, ou se este regime se aplicasse retroativamente,
o resultado seria substancialmente diferente,
pois o YouTube e a Uploaded seriam considerados como realizando
“atos de comunicação ao público”,
o que implicaria que teriam de comprovar que envidaram os melhores esforços
para obter autorizações dos titulares de direitos
e para impedir a disponibilização de obras não autorizadas,
sob pena de serem considerados responsáveis pelas infrações aos direitos de autor.

%---------------------------------------------------------------------------------------------------------------------------

\newpage
\renewcommand{\refname}{Bibliografia} % Para artigos
\renewcommand{\bibname}{Bibliografia} % Para livros e relatórios
\addcontentsline{toc}{section}{Bibliografia} % Adiciona a Bibliografia ao índice
\printbibliography

\end{document}
