\documentclass[a4paper]{article}
% Pacotes necessários
\usepackage[portuguese]{babel}
\usepackage[backend=biber, style=apa, citestyle=apa, language=portuguese]{biblatex}
\usepackage{csquotes}
\addbibresource{Recursos/referencias.bib}

\usepackage{amsmath}
\usepackage{graphicx}
\usepackage{subcaption}
\usepackage{setspace}
\usepackage{siunitx} % Required for alignment
\sisetup{
  round-mode          = places, % Rounds numbers
  round-precision     = 2, % to 2 places
}
\usepackage{enumerate}
\usepackage{enumitem}
\usepackage{amsmath}
\usepackage{karnaugh-map}
\usepackage[section]{placeins}
\usepackage{geometry}
\usepackage{amssymb}
\usepackage{titling}
\usepackage[T1]{fontenc}
\usepackage{float}
\usepackage[hidelinks]{hyperref}
\usepackage{xcolor}
\usepackage{indentfirst}
\usepackage{array}
\usepackage{soul}
\usepackage{afterpage}
\newcolumntype{P}[1]{>{\centering\arraybackslash}p{#1}}
\onehalfspacing

% Comando para criar uma página vazia
\newcommand\myemptypage{
    \null
    \thispagestyle{empty}
    \addtocounter{page}{-1}
    \newpage
}

% Página de título principal
\newcommand{\firsttitlepage}{
    \begin{titlepage}
        \centering
        \vspace*{1cm}
        
        % Logos superior
        \begin{figure}[h!]
            \centering
            \includegraphics[width=6cm]{Recursos/LOGO_IPB} % Substitua pelo caminho da imagem
            \vspace{0.5cm}
        \end{figure}

        % Informações da instituição
        \large\textbf{INSTITUTO POLITÉCNICO DE BEJA} \\
        \large\textbf{Escola Superior de Tecnologia e Gestão} \\
        \large\textbf{Mestrado em Engenharia de Segurança Informática} \\
        \large\textbf{Direito na Segurança Informática e no Cibercrime} \\
        
        \vspace{2cm}
        
        % Título do projeto
        {\Huge \textbf{Acórdão de 22 de junho de 2021}} \\
        
        \vspace{1.5cm}
        
        % Autores
        \large Martinho José Novo Caeiro - 23917 \\
        
        \vfill
        
        % Logo inferior
        \begin{figure}[h!]
            \centering
            \includegraphics[width=6cm]{Recursos/IPBejaESTIG.jpg} % Substitua pelo caminho da imagem
        \end{figure}
        
        \vspace{1cm}
        
        % Local e data
        {\large Beja, janeiro de 2026}
    \end{titlepage}
}

\newcommand{\secondtitlepage}{
    \begin{titlepage}
        \centering
        \vspace*{1cm}
        
        % Informações da instituição
        \large\textbf{INSTITUTO POLITÉCNICO DE BEJA} \\
        \large\textbf{Escola Superior de Tecnologia e Gestão} \\
        \large\textbf{Mestrado em Engenharia de Segurança Informática} \\
        \large\textbf{Direito na Segurança Informática e no Cibercrime} \\
        
        \vspace{2cm}
        
        % Título do projeto
        {\Huge \textbf{Acórdão de 22 de junho de 2021}} \\
        
        \vspace{1.5cm}
        
        % Autores
        \large Martinho José Novo Caeiro - 23917 \\

        \vspace{2cm}

        % Orientador
        \large Orientador: Manuel David Rodrigues Masseno \\
        
        \vfill
        
        % Local e data
        {\large Beja, janeiro de 2026}
    \end{titlepage}
}

\begin{document}


\pagenumbering{gobble} % Oculta numeração da página

% Primeira página de título
\firsttitlepage

\secondtitlepage

\vspace{1cm}
\newpage
%--------------------------------------------------------------------------------------------------------------------------------------

\vspace{1cm}

% Início do conteúdo do relatório
\newpage
\doublespacing
\doublespacing

\newpage
\pagenumbering{arabic}

\section{O Acórdão}\label{intro}
O acordão em questão trata-se de dois processos, C‑682/18, onde Frank Peterson, produtor de música, 
colocou o YouTube e a sua representante legal, a Google, nos tribunais alemães devido à distribuição
no YouTube, em 2008, de vários fonogramas sobre os quais, alegadamente, 
Peterson detém vários direitos (\cite{acordao20210622}).

E C‑683/18, onde Elsevier colocou Cyando nos tribunais alemães devido à distribuição 
na sua plataforma de armazenagem e de partilha de ficheiros «Uploaded», em 2013, 
de várias obras sobre as quais a Elsevier detém direitos exclusivos.

O Bundesgerichtshof (Supremo Tribunal Federal Alemão), tomou conhecimento destes dois processos e, 
submeteu várias questões prejudiciais ao Tribunal de Justiça para que este determine, 
a responsabilidade dos operadores de plataformas em linha quando estejam em causa obras protegidas 
pelos direitos de autor que são colocadas nessas plataformas, de forma ilícita, pelos seus utilizadores.

Nos próximos tópicos será feita uma análise do acordão, 
de modo a encontrar o porquê das decisões tomadas pelo Tribunal de Justiça.

%---------------------------------------------------------------------------------------------------------------------------
\newpage
\section{Análise Jurídica}\label{con}

%---------------------------------------------------------------------------------------------------------------------------
\subsection{Com atos da época}
A primeira pergunta feita pelo Bundesgerichtshof é se o operador de uma plataforma de partilha de vídeos 
ou de uma plataforma de armazenagem e de partilha de ficheiros, é responsável quando os utilizadores colocam 
ilegalmente à disposição do público conteúdos protegidos.\\

Assim sendo, temos o facto de que os autores tem direito a autorizar ou proibir a utilização das suas obras,
abrangido no Artigo 3.º da Diretiva 2001/29/CE (\cite{diretiva200129-1}), onde confere aos titulares de direitos 
o controlo sobre os atos de comunicação ao público, mas não impõe, por si só, uma obrigação geral direta aos operadores de plataformas,
sendo necessário apurar se estes praticam um ato de “comunicação ao público”.

Neste contexto, o Tribunal de Justiça decidiu que em regra, 
os operadores em questão não realizam um “ato de comunicação ao público” quando os utilizadores 
carregam conteúdos ilegais, salvo se tiver um papel ativo e deliberado na disponibilização desses conteúdos.

Não só, os autores tambem tem o direito a obter injunções contra os intermediários cujos serviços foram usados para violar
os direitos de autor conforme o Artigo 8.º/3 da mesma Diretiva, isto significa que se o autor descobrir que o operador
não está a respeitar o seu pedido pode recorrer aos tribunais para fazer valer os seus direitos.\\

Perante estes dois artigos conclui-se que os autores tem meios legais para proteger os seus direitos de autor
e para poderem agir judicialmente quando necessário.

\newpage
A segunda pergunta revolve em torno da possibilidade dos operadores tirarem proveito do facto de que, 
por norma não tem responsabilidade pelo conteudo colocado na plataforma.\\

Responde-se a esta pergunta com o Artigo 14.º da Diretiva 2000/31/CE (\cite{diretiva200031-1}), onde é referido que o operador beneficia 
de uma isenção de responsabilidade pelo conteúdo colocado pelos utilizadores, desde que não tenha conhecimento do conteudo ilegal,
ou que, aja rapidamente para remover ou bloquear o acesso ao conteudo ilegal assim que tenha conhecimento do mesmo.

Este critério de “conhecimento” não está isento de zonas cinzentas,
uma vez que inclui não apenas o conhecimento efetivo, mas também o conhecimento de factos ou circunstâncias
que tornem o ato ilegal.

Mas os operadores tem meios para se protegerem, como por exemplo, com o Artigo 15.º da mesma Diretiva,
onde é referido que os operadores não são obrigados a monitorizar o conteudo colocado pelos utilizadores,
a não ser que exista uma ordem judicial nesse sentido.\\

Perante estes dois artigos conclui-se que os operadores não são automaticamente responsaveis pelo conteudo e tambem 
não são obrigados a ser proativos em relação ao conteudo ilegal, mas devem agir rapidamente 
quando tenham conhecimento do conteudo ilegal.\\

Dados todos os factos, conclui-se que mesmo sem a obrigatoriedade dos operadores de monitorizarem o conteudo,
os autores tem todo o direito de pedir a remoção do conteudo ilegal e de recorrer a injunções judiciais
dado que estas diretivas não entram em conflito uma com a outra.

%---------------------------------------------------------------------------------------------------------------------------
\newpage
\subsection{Com atos correntes}
Ambas Diretivas tiveram alterações, mais especificamente em 2017 e 2019 para a Diretiva 2001/29/CE (\cite{diretiva200129-2}) 
e em 2024 para a Diretiva 2000/31/CE (\cite{diretiva200031-2}).\\

Em especial, temos a Diretiva 2019/790 (\cite{diretiva2019790}) que introduziu no seu Artigo 17.º,
a possibilidade de considerar que determinados prestadores de serviços de partilha de conteúdos em linha
realizam por vontade propria, "atos de comunicação ao público" ou de colocação à disposição do público
quando disponibilizam conteúdos carregados pelos seus utilizadores.

Consequentemente, estas plataformas passam a ser, diretamente responsáveis
pelos conteúdos protegidos por direitos de autor,
salvo se demonstrarem os melhores esforços para obter autorizações dos titulares de direitos
e para impedir a disponibilização de obras não autorizadas.\\

No entanto, este novo regime não é aplicável retroativamente, isto significa que os processos
C-682/18 e C-683/18 não irão ter a sua decisão alterada.

Assim, apesar de a Diretiva 2019/790 alterar profundamente o regime de responsabilidade das plataformas
para o futuro, tal não teve impacto direto na solução jurídica adotada pelo Tribunal de Justiça
no acórdão em análise, mantendo-se válidas, para estes casos concretos,
as conclusões retiradas anteriormente.\\

Caso estes processos fossem julgados atualmente ou o regime se aplicasse retroativamente, o resultado seria completamente diferente,
pois o Youtube e o Uploaded seriam considerados que estariam a realizar "atos de comunicação ao público", isto significa que teriam
de comprovar que fizeram os melhores esforços para obter autorizações dos titulares de direitos
e para impedir a disponibilização de obras não autorizadas.

%---------------------------------------------------------------------------------------------------------------------------

\newpage
\renewcommand{\refname}{Bibliografia} % Para artigos
\renewcommand{\bibname}{Bibliografia} % Para livros e relatórios
\addcontentsline{toc}{section}{Bibliografia} % Adiciona a Bibliografia ao índice
\printbibliography

\end{document}
